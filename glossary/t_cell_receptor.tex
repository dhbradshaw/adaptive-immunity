\section{T-cell Receptor}
\label{Glossary:T_cell_receptor}
Some white blood cells have the ability to recognize pathogens.  They do so via
a membrane-bound protein complex known as the T-cell Receptor.  White blood cells
that have T-cell Receptors are called T-cells.

The T-cell Receptor is in the immunoglobulin protein superfamily.  Like antibodies,
it has variability through somatic mutation and is used for antigen recognition.

The T-cell Receptor does not undergo somatic hypermutation.
