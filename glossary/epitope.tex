\section{Epitope}
\label{Glossary:Epitope}

The recognized portion of an antigen.  Also called an \textit{anitgenic determinant}.

A given antigen may have many epitopes corresponding to many different antibodies, T-cells, or B-cells.

If an epitope is covered or altered, its corresponding recognizer may loose it's ability to recognize the antigen.

\subsection{Epitope Size}
A typical peptide epitope is 6-8 amino acids long.
Assuming a per-amino-acid length of 3.5 angstroms, that means that a typical epitope is from
\textbf{2.1 to 2.8 nm} in length if it's stretched out in a line.
\subsection{Epitope Degeneracy}
Considering only amino acid sequence differences (and therefore not, for example
tertiary structure or glycosylation related effects) and leaving out selenocysteine,
a length of 8 amino acids could have \textbf{9e18} different sequences.
